\documentclass[12pt,english,a4paper]{article}
\usepackage[utf8]{inputenc}
\usepackage[T1]{fontenc}
\usepackage{babel,amsmath,amsthm,graphicx,mathtools,textcomp,varioref,amssymb,float,listings}
\usepackage[top=50pt,left=40pt,right=40pt]{geometry}
\usepackage[format=plain,labelfont=bf,up,font=small]{caption}
\usepackage{titling,wrapfig}
\usepackage{color}
\usepackage{csquotes}
\usepackage{verbatim}
\usepackage{csvsimple}
\usepackage{booktabs}
\usepackage{indentfirst}
\usepackage{chngcntr}
\counterwithin{table}{subsection}
\counterwithin{figure}{subsection}
\usepackage[
backend=biber,
style=alphabetic,
citestyle=numeric,
sorting=nyt
]{biblatex}

\addbibresource{bibliography.bib}

\DeclarePairedDelimiter\abs{\lvert}{\rvert}%
\DeclarePairedDelimiter\norm{\lVert}{\rVert}%

\tolerance = 5000
\hbadness = \tolerance
\pretolerance = 2000

\setlength{\headheight}{15pt}
\setlength{\parskip}{1em}

\usepackage{fancyhdr}
    \pagestyle{fancy}
    \fancyhf{}
    \fancyhead[R]{Variational Monte Carlo}
    \fancyhead[C]{}
    \fancyhead[L]{\leftmark}
    \fancyfoot[L]{Henrik Lind Petlund}
    \fancyfoot[C]{}
    \fancyfoot[R]{Page \thepage}
    \renewcommand{\headrulewidth}{.5pt}
    \renewcommand{\footrulewidth}{.5pt}
    \usepackage{url}

\title{Title here}
\author{Henrik Lind Petlund}

\begin{document}
\begin{titlepage}
\maketitle
\begin{abstract}

\newpage

\end{abstract}
\tableofcontents
\end{titlepage}

\section{Introduction} \label{section:introduction}
This report will study a system of \textit{quantum dots} in three dimensions. The system will be consisting of $N=2$ electrons restricted to move in potential traps of the harmonic oscillator type. Analytical values for the energy for specific HO frequencies is found in the article \cite{Taut} by M. Taut. The aim of this report is to study the above mentioned system using the Variational Monte Carlo (VMC) method and and discuss aspects of correlations due to the electron-electron repulsive interactions. The ground state energy, relative distance between two electrons, and the potential and kinetic energies of the quantum dots system will also be evaluated.

The system at hand consists of two electrons in a three dimensional harmonic oscillator potential, where the general expression of the Hamiltonian is
\begin{equation}
    \hat H = \sum_{i=1}^N \left(-\frac{1}{2}\nabla_i^2 +\frac{1}{2}\omega^2r_i^2\right)+\sum_{i<j}\frac{1}{r_{ji}}
    \label{eq:hamiltonian_total}
\end{equation}
where $r_{ij}=\sqrt{\boldsymbol{r_1}-\boldsymbol{r_2}}$ is the distance between the two electrons and $r_i=\sqrt{x_i^2+y_i^2+z_i^2}$ is the modulus of the position of a given electron $i$. Natural units are used and all energies are in terms of atomic units, a.u., which in turn makes all distances, $r$, dimensionless. The unperturbed part of the Hamiltonian is given by the harmonic oscillator part as
\begin{equation}
    \hat H^0 = \sum_{i=1}^N \left(-\frac{1}{2}\nabla_i^2 +\frac{1}{2}\omega^2r_i^2\right)
    \label{eq:hamiltonian_unperturbed}
\end{equation}
while the perturbed part is given by the repulsive interactions of the two electrons as
\begin{equation}
    \hat H^1 = \sum_{i<j}\frac{1}{r_{ji}}
    \label{eq:hamiltonian_perturbed}
\end{equation}

Firstly, the simple unperturbed system will be described to find some useful benchmarks for further expansion of the problem. Thereafter, the analytic energies for two trail wave functions will be derived and compared to the ones calculated in \textit{"Project 2: Quantum dots in three dimensions"} \cite{GitHub} prior this fall. Finally, the Variational Monte Carlo method together with the trail wave functions will be used to estimate the ground state energy and eventually test the \textit{virial theorem}. 

\newpage

\section{Methods and theory} \label{section:theory}
\subsection{The non-interacting case} \label{section:theory:non-interacting}

In the non-interacting case, the hamiltonian is the unperturbed one from Equation \eqref{eq:hamiltonian_unperturbed}. For a single electron system, the wave function may be described with Hermite polynomials as
\begin{equation}
    \phi_{n_x,n_y,n_z}(x,y,z)=AH_{n_x}(\sqrt(\omega x))H_{n_y}(\sqrt(\omega y))H_{n_z}(\sqrt(\omega z))e^{-\frac{\omega (x^2+x^2+z^2)}{2}}
\end{equation}
In the ground state ($n_x,n_y,x_z=0$), we have the single electron wave function as
\begin{equation}
    \phi_{GS}(\boldsymbol{r})=Ae^{-\frac{1}{2}\omega r^2}
\end{equation}
where $r_i$ is defined in Section \ref{section:introduction}. Operating with the unperturbed Hamiltonian on $\phi_{GS}(\boldsymbol{r_1})$ and using the fact that
\begin{align*}
    -\frac{1}{2}\nabla^2 e^{-\frac{1}{2}\omega (x^2+y^2+z^2)}=\left(-\frac{1}{2}\omega^2\left(x^2+y^2+z^2\right)+\frac{1}{2}\omega+\frac{1}{2}\omega+\frac{1}{2}\omega\right)e^{-\frac{1}{2}\omega (x^2+y^2+z^2)}
\end{align*}
\begin{equation}
    \rightarrow -\frac{1}{2}\nabla^2 e^{-\frac{1}{2}\omega (x^2+y^2+z^2)}=\left(-\frac{1}{2}\omega^2r^2+\frac{3}{2}\omega\right)e^{-\frac{1}{2}\omega r^2}
    \label{eq:partial_derivative}
\end{equation}
results in
\begin{align*}
    H\phi_{GS}(\boldsymbol{r})=\frac{3}{2}\omega\phi_{GS}(\boldsymbol{r})
\end{align*}
which in terms means that adding another electron to the system would result in a contribution of another $3/2\omega$ to the ground state energy. Resulting in the lowest lying state energy $3\omega$, which we note as a benchmark for later calculations. The wave function for the ground state of two electrons would then be
\begin{equation}
    \Phi(\boldsymbol{r_1},\boldsymbol{r_2})=Ce^{-\frac{1}{2}\omega(r_1^2+r_2^2)}
    \label{eq:groundstate_two_electrons}
\end{equation}

To be sure that this wave function actually is the ground state, let's examine the radial Schroedingers equation for one electron in a harmonic potential
\begin{equation}
    -\frac{\hbar^2}{2m}\left(\frac{1}{r^2}\frac{d^2}{dr^2}-\frac{l\left(l+1\right)}{r^2}\right)R(r)+V(r)R(r)=ER(r)
    \label{eq:schroedinger_ho_pot}
\end{equation}
where the solution is
\begin{equation}
    E_{nl}=\hbar\omega\left(2n+l+\frac{3}{2}\right)
\end{equation}
From this, it's convenient to find the lowest energy when $n=0$ and $l=0$ to be $\frac{3}{2}\omega$. This does in turn confirm that Equation \eqref{eq:groundstate_two_electrons} is the ground state wave function of the two electron system.

Since the system at hand is a two-body system, we must combine the spin of the two electrons (addition of angular Momenta may be found in \cite{Griffiths} Section 4.4.3). Since electrons are fermions, they are spin-1/2 particles. Combining two spin-1/2 particles results in either a total spin of $S=0$ (singlet) or $S=1$ (triplet), which further gives four possible eigenstates (1 and 3 respectively). $S=1$ refers to three symmetric wave functions, whereas $S=0$ refers to one antisymmetric wave function in terms of particle change. Of these, the system with $S=0$ has the lowest energy, and is therefore the total spin of the ground state wave function in Equation \eqref{eq:groundstate_two_electrons}. This can also be explained from the fact that the \textit{centrifugal barrier} in Equation \eqref{eq:schroedinger_ho_pot} becomes $0$ for $l=0$ with one electron, and gives in a similar way the lowest energy for $S=0$ in the two-electron system.

\subsection{The trail wave functions and their energies} \label{section:theory:energies}

The two trail wave functions to study are
\begin{equation}
    \Psi_{T1}(\boldsymbol{r_1},\boldsymbol{r_2})=Ce^{-\frac{1}{2}\alpha\omega(r_1^2+r_2^2)}
    \label{eq:trail_one}
\end{equation}
\begin{equation}
    \Psi_{T2}(\boldsymbol{r_1},\boldsymbol{r_2})=Ce^{-\frac{1}{2}\alpha\omega(r_1^2+r_2^2)}e^{\frac{r_{12}}{2(1+\beta r_{12})}}
    \label{eq:trail_two}
\end{equation}
where $\alpha$ and $\beta$ are the variational parameters.

Now, since the problem at hand involves Coulomb (electron-electron) interactions, it is important to see if the trail wave functions satisfy the two-body cusp condition by T. Kato \cite{kato_cusp}. This cusp condition makes sure that one eliminates all possible singularities that may arise due to the the $r$ terms in the wave function. The two-body cusp condition reads in short
\begin{equation}
    \left[\frac{d\Psi_{Ti}}{dr_1}\right]_{r_1\rightarrow 0}=-Z\Psi(0,r_2,r_{12})
    \label{cusp_r1}
\end{equation}
\begin{equation}
    \left[\frac{d\Psi_{Ti}}{dr_2}\right]_{r_2\rightarrow 0}=-Z\Psi(r_1,0,r_{12})
    \label{cusp_r2}
\end{equation}
\begin{equation}
    \left[\frac{d\Psi_{Ti}}{dr_{12}}\right]_{r_{12}\rightarrow 0}=\frac{1}{2}\Psi(\frac{1}{2}(r_1+r_2),\frac{1}{2}(r_1+r_2),0)
    \label{cusp_r12}
\end{equation}
where $Z=1$ is the nuclear charge. For the first trail wave function we only have to evaluate the first two equations, since there is no $r_{12}$ term. As for the second trail wave function, there are two $r_{12}$ terms, and all three equations must hold. These conditions are shown to be satisfied in Section \ref{section:appendix:calculations:cusp}, and we should therefore not expect any singularity issues.

The local energy is given by the expression
\begin{equation}
    E_L(\boldsymbol{R},\boldsymbol{\alpha})=\frac{1}{\psi_T(\boldsymbol{R},\boldsymbol{\alpha})}H\psi_T(\boldsymbol{R},\boldsymbol{\alpha})
    \label{eq:local_energy}
\end{equation}
where $\boldsymbol{R}=(\boldsymbol{r_1},\boldsymbol{r_2})$. Using Equation \eqref{eq:partial_derivative} for the partial derivative, the analytical local energy for the first trail wave function becomes
\begin{align*}
    E_{L1}=\frac{1}{\psi_{T1}(\boldsymbol{R},\boldsymbol{\alpha})}H^0\psi_{T1}(\boldsymbol{R},\boldsymbol{\alpha})
\end{align*}
\begin{align*}
    =\left(-\frac{1}{2}\alpha^2\omega^2r_1^2-\frac{1}{2}\alpha^2\omega^2r_2^2+\frac{1}{2}\omega^2\left(r_1^2+r_2^2\right)+\frac{3}{2}\omega +\frac{3}{2}\omega\right)
\end{align*}
\begin{equation}
    \rightarrow E_{L1}=-\frac{1}{2}\omega^2\left(r_1^2+r_2^2\right)\left(1-\alpha^2\right)+3\omega
    \label{eq:local_energy_t1}
\end{equation}
where Coulomb interactions has been excluded. 

Adding the Coulomb interaction gives the local energy
\begin{equation}
    E_{L1}=-\frac{1}{2}\omega^2\left(r_1^2+r_2^2\right)\left(1-\alpha^2\right)+3\omega+\frac{1}{r_{12}}
    \label{eq:local_energy_coulomb_t1}
\end{equation}
Whereas the second trail wave function with the extra exponential term has the local energy
\begin{equation}
    E_{L2}=E_{L1} + \frac{1}{2\left(1+\beta r_{12}\right)^2}\left[\alpha\omega r_{12}-\frac{1}{2\left(1+\beta r_{12}\right)^2}-\frac{2}{r_{12}}+\frac{2\beta}{1+\beta r_{12}}\right]
    \label{eq:local_energy_coulomb_t2}
\end{equation}
which is derived in Section \ref{section:appendix:calculations:energies}. It is evidently these two expressions for the local energy of the trail wave functions that are to be computed, and thereafter compared to each other and the exact solution of the ground state.

The exact value for the ground state energy is given in \cite{proj5} as $3.556$ a.u. for $\omega = 1$. In Table \ref{tab:proj2}, the analytic eigenvalues calculated by Taut \cite{Taut} is given, accompanied by the experimental values calculated in \textit{Project 2} \cite{GitHub}. The values are fetched from Table 4.2.1 and 6.1.1 in \textit{Project 2}. It is important to note that the values in Table \ref{tab:proj2} gives the eigenvalues as function of the relative HO frequency $\omega_r$, where $\omega = 2\omega_r$ \cite{Taut}. The energies in \ref{tab:proj2} are also the relative energies, and as stated in \cite{proj5} - one needs to add $1.5$ a.u. to the relative energy values (because of the center-of-mass energy). It's easy to see that this gives in fact $3.5566$ for the analytic  $\omega = 1$ ($\omega_r =0.5$) case, whereas the numerical approximation with the Töplitz-matrix from \textit{Project 2} has a deviation of $\sim 0.18$ a.u.

\subsection{The Variational Monte Carlo method (VMC)} \label{section:theory:VMC}
The Variational Monte Carlo method (hereafter; VMC) is based on the Rayleigh-Ritz variational principle. This principle states that if one chooses any trail wave function $\psi_{b}$, where $b$ is a variational parameter, then the energy
\begin{align*}
    E_{Tr}=\frac{\langle\psi_{b}|H|\psi_{b}\rangle}{\langle\psi_{b}|\psi_{b}\rangle}\le E_{gs}
\end{align*}
is an upper bound on the ground state energy of the system. Minimizing $E_Tr$ with respect to $b$ gives then the lowest upper bound within the class of trail states. The VMC method uses this principle to vary one or more variational parameter numerically, in hope to find the lowest energy of the system. This is the variational part of the method. 

As for the Monte Carlo part of the method - which is run for every change in variational parameter - we define a number of MC-cycles, initialize a position, $\boldsymbol{R}$, the energy and the variance. Then we calculate the probability density
\begin{equation}
    P(\boldsymbol{R})=|\Psi_{Tr}^\alpha(\boldsymbol{R})|^2
    \label{eq:probalility_density}
\end{equation}
Thereafter, we generate a random variable $r\in [0,1]$ and calculate the trail position
\begin{align*}
    \boldsymbol{R_p}=\boldsymbol{R}+r\cdot\delta
\end{align*}
where $\delta$ is a chosen step size, which is chosen such that approximately $50\%$ of the trail positions are accepted. Further, the Metropolis algorithm is deployed to accept or reject the chosen trail position. This is done by calculating
\begin{align*}
    w=\frac{P(\boldsymbol{R_p})}{P(\boldsymbol{R})}
\end{align*}
and generate a random number $t\in[0,1]$. If $t\leq w$, the new position is accepted, and otherwise rejected. Lastly, we either calculate new mean values, or keep the old ones. Since there are two electrons in the system, the above mentioned procedure is done twice. Hence; one MC-cycle inhibits two of the above procedures.

\subsection{The Virial Theorem} \label{section:theory:virial}

The Virial Theorem states that there is a proportionality between the expectation value of the total kinetic energy, $\langle T\rangle$, and the total potential energy, $\langle V\rangle$. For a pure harmonic oscillator, this proportonality is given
\begin{equation}
    \langle T\rangle = \langle V\rangle
    \label{virial_theorem}
\end{equation}
By plotting the ratio $\langle T\rangle / \langle V\rangle$ as function of the HO frequency $\omega$, one can study the validity of the Virial Theorem. If valid; the ratio should be a constant.

\section{Results} \label{section:results}

\subsection{Stability} \label{section:results:stability}

- Plot stability of energy and variance as function of MC-cycles.

\subsection{Optimal parameters and expectation values} \label{section:results:optimal}

Table with optimal alpha abd beta values and the expectation values for different omegas.

\begin{table}[htb]
    \centering
    \csvreader[tabular=lll,
        head=false,
        table head=\toprule,
        late after line=\\,
        late after first line=\\\midrule,
        table foot=\bottomrule,
        ]{../data/expectation_values.csv}{}{\csvlinetotablerow}
        \caption{Expectation}
        \label{tab:expectation_values}
\end{table}

\subsection{The Virial Theorem} \label{section:results:virial}

Figure \ref{fig:virial_trail_1} and \ref{fig:virial_trail_2} below presents the validity of the Virial Theorem for both trail wave functions with and without the electron-electron interactions.

\begin{figure}[H]
    \centering
    \includegraphics[scale=0.7]{../figures/plot_virial_trail_1.pdf}
    \caption{Plot of the ratio $\langle T\rangle / \langle V\rangle$ as function of HO frequency $\omega$ for $\Psi_{T1}$ with and without the repulsive electron-electron interactions. The optimal parameter for $\alpha$ is used.}
    \label{fig:virial_trail_1}
\end{figure}
\begin{figure}[H]
    \centering
    \includegraphics[scale=0.7]{../figures/plot_virial_trail_2.pdf}
    \caption{Plot of the ratio $\langle T\rangle / \langle V\rangle$ as function of HO frequency $\omega$ for $\Psi_{T2}$ with and without the repulsive electron-electron interactions. The optimal parameter for $\alpha$ and $\beta$ are used.}
    \label{fig:virial_trail_2}
\end{figure}


\section{Discussion} \label{section:discussion}

\subsection{Stability} \label{section:discussion:stability}

\subsection{Optimal parameters and expectation values} \label{section:discussion:optimal}

\subsection{The Virial Theorem} \label{section:discussion:virial}

\section{Conclusion} \label{section:conclusion}

\section{Appendix} \label{section:appendix}
\subsection{Calculations} \label{section:appendix:calculations}
\subsubsection{Cusp conditions} \label{section:appendix:calculations:cusp}
As we already know that the cusp conditions are satisfied for the single electron (hydrogen) wave function, we do not need to evaluate the first trail wave function in Equation \eqref{eq:trail_one}, as it is a combination of this one. The second trail wave function in Equation \eqref{eq:trail_two} has a $r_{12}$ term, and we therefore need to evaluate if the condition
\begin{align*}
    \left[\frac{d\Psi_{Ti}}{dr_{12}}\right]_{r_{12}\rightarrow 0}=\frac{1}{2}\Psi(\frac{1}{2}(r_1+r_2),\frac{1}{2}(r_1+r_2),0)
\end{align*}
holds.

Calculating
\begin{align*}
    \left[\frac{d\Psi_{T2}}{dr_{12}}\right]=\left(\frac{2\left(1+\beta r_{12}\right)-r_{12}2\beta}{4\left(1+\beta r_{12}\right)^2}\right)\Psi_{T2}
\end{align*}
and as $r_{12}\rightarrow 0$ we have
\begin{align*}
    \left[\frac{d\Psi_{T2}}{dr_{12}}\right]_{r_{12}\rightarrow 0}=\frac{1}{2}e^0=\frac{1}{2}\Psi_{T2}(\frac{1}{2}(r_1+r_2),\frac{1}{2}(r_1+r_2),0)
\end{align*}
Which in terms means that the cusp condition is satisfied for both trail wave functions.

\subsubsection{Local energy of second trail wave functions} \label{section:appendix:calculations:energies}
First off, we simplify the expression for the second trail wave function as (excluding the normalization constant $C$)
\begin{align*}
    \Psi_{T2}=Ce^{-\frac{1}{2}\alpha\omega(r_1^2+r_2^2)}e^{\frac{r_{12}}{2(1+\beta r_{12})}}=A\cdot B
\end{align*}
where $A$ and $B$ are the respective exponential expressions and $B=e^{\frac{|r_1-r_2|}{2(1+\beta |r_1-r_2|)}}$. Acting with the squared del operator on $A\cdot B$ gives then
\begin{equation}
    \nabla_i^2\Psi_{T2}=\left(B\nabla_i^2A+2\nabla_iA\nabla_iB+A\nabla_i^2B\right)
    \label{eq:double_deriv}
\end{equation}
We calculate how the del operator acts on $A$
\begin{align*}
    \nabla_iA = -\alpha\omega \left(x_i+y_i+z_i\right)A
\end{align*}
\begin{align*}
    \nabla_i^2A = \left(\frac{d^2}{dx_i^2}+\frac{d^2}{dy_i^2}+\frac{d^2}{dz_i^2}\right)A
\end{align*}
\begin{align*}
    = \left[-\alpha\omega+\alpha^2\omega^2x_i^2-\alpha\omega+\alpha^2\omega^2y_i^2-\alpha\omega+\alpha^2\omega^2z_i^2\right]A
\end{align*}
\begin{align*}
    = \left[\alpha^2\omega^2\left(x_i^2+y_i^2+z_i^2\right)-3\alpha\omega\right]A
\end{align*}
\begin{align*}
    \rightarrow \nabla_i^2A = \left[\alpha^2\omega^2r_i^2-3\alpha\omega\right]A
\end{align*}
Using that $\frac{dr_{12}}{dx_i}=\frac{\left(x_i-x_{j\ne i}\right)}{r_{12}}$ we can find the expressions for how the del operator acts on $B$. We do this for just the $d/dx_i$-part first.
\begin{align*}
    \frac{d}{dx_i}B = \frac{\left(x_i-x_{j\ne i}\right)}{r_{12}\cdot 2\left(1+\beta r_{12}\right)^2}B
\end{align*}
\begin{align*}
    \frac{d^2}{dx_i^2} B = \left[\frac{r_{12}2\left(1+\beta r_{12}\right)^2}{r_{12}^2 4\left(1+\beta r_{12}\right)^4}-\frac{\left(x_i-x_{j\ne 1}\right)\left(\frac{\left(x_i-x_{j\ne 1}\right)}{r_{12}}2\left(1+\beta r_{12}\right)^2+r_{12}4\left(1+\beta r_{12}\right)\frac{\beta\left(x_i-x_{j\ne 1}\right)}{r_{12}}\right)}{r_{12}^2 4\left(1+\beta r_{12}\right)^4}\right]B\\
    +\left[\left(\frac{\left(x_i-x_{j\ne i}\right)}{r_{12}\cdot 2\left(1+\beta r_{12}\right)^2}\right)^2\right]B
\end{align*}
Defining
\begin{align*}
    F_i^{(x)}=\frac{\left(x_i-x_{j\ne i}\right)^2}{r_{12}^2}
\end{align*}
where $\left(F_i^{(x)}+F_i^{(y)}+F_i^{(z)}\right)=1$. We can write
\begin{align*}
    \frac{d^2}{dx_i^2} B = \left[\frac{1}{r_{12}2\left(1+\beta r_{12}\right)^2}- F_i^{(x)}\frac{1}{r_{12}2\left(1+\beta r_{12}\right)^2}-F_i^{(x)}\frac{2\beta}{2\left(1+\beta r_{12}\right)^3}+F_i^{(x)}\frac{1}{4\left(1+\beta r_{12}\right)^4}\right]B
\end{align*}
So that the del operators on $B$ gives
\begin{align*}
    \nabla_i B = \frac{\left(x_i-x_{j\ne i}\right)+\left(y_i-y_{j\ne i}\right)+\left(z_i-z_{j\ne i}\right)}{r_{12}\cdot 2\left(1+\beta r_{12}\right)^2}B
\end{align*}
and
\begin{align*}
    \nabla_i^2 B = \left[3\cdot\frac{1}{r_{12}2\left(1+\beta r_{12}\right)^2}-\left(1\right)\cdot\frac{1}{r_{12}2\left(1+\beta r_{12}\right)^2}-\left(1\right)\cdot\frac{2\beta}{2\left(1+\beta r_{12}\right)^3}+\left(1\right)\cdot\frac{1}{4\left(1+\beta r_{12}\right)^4}\right]B
\end{align*}
\begin{align*}
    = \frac{1}{2\left(1+\beta r_{12}\right)^2}\left[\frac{2}{r_{12}}-\frac{2\beta}{1+\beta r_{12}}+\frac{1}{2\left(1+\beta r_{12}\right)^2}\right]
\end{align*}
Precalculating the expression
\begin{align*}
    2\nabla A_1\nabla B_1+2\nabla A_2\nabla B_2
\end{align*}
\begin{align*}
    =-\frac{2\alpha\omega}{2\left(1+\beta r_{12}\right)^2}\left(\frac{x_1\left(x_1-x_2\right)+x_2\left(x_2-x_1\right)+y_1\left(y_1-y_2\right)+y_2\left(y_2-y_1\right)+z_1\left(z_1-z_2\right)+z_2\left(z_2-z_1\right)}{r_{12}}\right)
\end{align*}
\begin{align*}
    = -\frac{2\alpha\omega}{2\left(1+\beta r_{12}\right)^2}\left(\frac{x_1^2+y_1^2+z_1^2-2\left(x_1x_2+y_1y_2+z_1z_2\right)+x_2^2+y_2^2+z_2^2}{r_{12}}\right)
\end{align*}
\begin{align*}
    = -\frac{2\alpha\omega}{2\left(1+\beta r_{12}\right)^2}\left(\frac{\boldsymbol{r_1}\cdot\boldsymbol{r_1}-2\boldsymbol{r_1}\cdot\boldsymbol{r_2}+\boldsymbol{r_2}\cdot\boldsymbol{r_2}}{r_{12}}\right)
\end{align*}
\begin{align*}
    = -\frac{2\alpha\omega}{2\left(1+\beta r_{12}\right)^2}\left(\frac{|\boldsymbol{r_1}-\boldsymbol{r_2}|^2}{r_{12}}\right)
\end{align*}
\begin{align*}
    = -\frac{2\alpha\omega}{2\left(1+\beta r_{12}\right)^2}r_{12}
\end{align*}
Now, writing the squared del operator acting on the second trail wave function as
\begin{align*}
    \nabla_i^2 \Psi_{T2}=D_i\Psi_{T2}
\end{align*}
where
\begin{align*}
    D_i=\left[\alpha^2\omega^2 r_i^2-3\alpha\omega\right]+2\nabla_i A\nabla_i B+\frac{1}{2\left(1+\beta r_{12}\right)^2}\left[\frac{2}{r_{12}}+\frac{1}{2\left(1+\beta r_{12}\right)^2}-\frac{2\beta}{1+\beta r_{12}}\right]
\end{align*}
Using the total Hamiltonian from Equation \eqref{eq:hamiltonian_total}, we may now get the expression for the local energy of the second trail wave function with the use of Equation \eqref{eq:local_energy} as
\begin{align*}
    E_{L2}=-\frac{1}{2}D_1-\frac{1}{2}D_2+\frac{1}{2}\omega^2r_1^2+\frac{1}{2}\omega^2r_2^2+\left(\frac{1}{r_{12}}\right)
\end{align*} 
\begin{align*}
    =E_{L1}-\left(-\frac{1}{2}\right)\frac{2}{2\left(1+\beta r_{12}\right)^2}\alpha\omega r_{12}-\left(\frac{1}{2}\right)\cdot 2\frac{1}{2\left(1+\beta r_{12}\right)^2}\left[\frac{2}{r_{12}}+\frac{1}{2\left(1+\beta r_{12}\right)^2}-\frac{2\beta}{1+\beta r_{12}}\right]
\end{align*}
\begin{align*}
    =E_{L1}+\frac{1}{2\left(1+\beta r_{12}\right)^2}\left[\alpha\omega r_{12}-\frac{1}{2\left(1+\beta r_{12}\right)^2}-\frac{2}{r_{12}}+\frac{2\beta}{1+\beta r_{12}}\right]
\end{align*}
where $E_{L1}$ is as in Equation \eqref{eq:local_energy_coulomb_t1} derived in Section \ref{section:theory:energies}.  

\subsection{Tables and data} \label{section:appendix:tables}

\begin{table}[htb]
    \centering
    \csvreader[tabular=lll,
        head=false,
        table head=\toprule,
        late after line=\\,
        late after first line=\\\midrule,
        table foot=\bottomrule,
        ]{../data/proj2.csv}{}{\csvcoli & \csvcolii & \csvcoliii}
        \caption{Experimental eigenvalues of the 2-electron system calculated in \textit{Project 2} and the analytic eigenvalues calculated using Eq. (16a) in \cite{Taut}. All as function of $\omega_r$. Here, $\omega_r=0.5$ implies $\omega=1$.}
        \label{tab:proj2}
\end{table}

\printbibliography

\end{document}

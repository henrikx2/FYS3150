\documentclass[12pt,english,a4paper]{article}
\usepackage[utf8]{inputenc}
\usepackage[T1]{fontenc}
\usepackage{babel,amsmath,amsthm,graphicx,mathtools,textcomp,varioref,amssymb,float,listings}
\usepackage[top=50pt,left=40pt,right=40pt]{geometry}
\usepackage[format=plain,labelfont=bf,up,textfont=it,up]{caption}
\usepackage{titling,wrapfig}
\usepackage{color}
\usepackage{csquotes}
\usepackage{verbatim}
\usepackage{csvsimple}
\usepackage{booktabs}
\usepackage[
backend=biber,
style=alphabetic,
citestyle=nature,
sorting=nyt
]{biblatex}

\addbibresource{bibliography.bib}

\newcommand*{\figuretitle}[1]{%
    {\centering%   <--------  will only affect the title because of the grouping (by the
    \textbf{#1}%              braces before \centering and behind \medskip). If you remove
    \par}%            these braces the whole body of a {figure} env will be centered.
}

\DeclarePairedDelimiter\abs{\lvert}{\rvert}%
\DeclarePairedDelimiter\norm{\lVert}{\rVert}%

\makeatletter
\csvset{
  autobooktabularcenter/.style={
    file=#1,
    after head=\csv@pretable\begin{tabular}{*{\csv@columncount}{l}}\csv@tablehead,
    table head=\toprule\csvlinetotablerow\\\midrule,
    late after line=\\\midrule,
    table foot=\\\bottomrule,
    late after last line=\csv@tablefoot\end{tabular}\csv@posttable,
    command=\csvlinetotablerow},
}
\makeatother
\newcommand{\csvautotabularcenter}[2][]{\csvloop{autotabularcenter={#2},#1}}
\newcommand{\csvautobooktabularcenter}[2][]{\csvloop{autobooktabularcenter={#2},#1}}

\tolerance = 5000
\hbadness = \tolerance
\pretolerance = 2000

\setlength{\headheight}{15pt}
\setlength{\parskip}{1em}

\usepackage{fancyhdr}
    \pagestyle{fancy}
    \fancyhf{}
    \fancyhead[R]{Variational Monte Carlo}
    \fancyhead[C]{}
    \fancyhead[L]{\leftmark}
    \fancyfoot[L]{Henrik Lind Petlund}
    \fancyfoot[C]{}
    \fancyfoot[R]{Page \thepage}
    \renewcommand{\headrulewidth}{.5pt}
    \renewcommand{\footrulewidth}{.5pt}
    \usepackage{url}

\title{Title here}
\author{Henrik Lind Petlund}

\begin{document}
\begin{titlepage}
\maketitle
\begin{abstract}

\newpage

\end{abstract}
\tableofcontents
\end{titlepage}

\section{Introduction} \label{introduction}

This report will study a system of \textit{quantum dots} in three dimensions. The system will be consisting of $N=2$ electrons restricted to move in potential traps of the harmonic oscillator type. Analytical values for the energy for specific HO frequencies is found in the article \cite{Taut} by M. Taut. The aim of this report is to study the above mentioned system using the Variational Monte Carlo (VMC) method and and discuss aspects of correlations due to the electron-electron repulsive interactions. The ground state energy, relative distance between two electrons, and the potential and kinetic energies of the quantum dots system will also be evaluated.

The system at hand consists of two electrons in a three dimensional harmonic oscillator potential, where the general expression of the Hamiltonian is
\begin{equation}
    \hat H = \sum_{i=1}^N \left(-\frac{1}{2}\nabla_i^2 +\frac{1}{2}\omega^2r_i^2\right)+\sum_{i<j}\frac{1}{r_ji}
    \label{eq:hamiltonian_total}
\end{equation}
where $r_{ij}=\sqrt{\boldsymbol{r_1}-\boldsymbol{r_2}}$ is the distance between the two electrons and $r_i=\sqrt{x_i^2+y_i^2+z_i^2}$ is the modulus of the position of a given electron $i$. Natural units are used and all energies are in terms of atomic units, a.u., which in turn makes all distances, $r$, dimensionless. The unperturbed part of the Hamiltonian is given by the harmonic oscillator part as
\begin{equation}
    \hat H^0 = \sum_{i=1}^N \left(-\frac{1}{2}\nabla_i^2 +\frac{1}{2}\omega^2r_i^2\right)
    \label{eq:hamiltonian_unperturbed}
\end{equation}
while the perturbed part is given by the repulsive interactions of the two electrons as
\begin{equation}
    \hat H^1 = \sum_{i<j}\frac{1}{r_ji}
    \label{eq:hamiltonian_perturbed}
\end{equation}

Firstly, the unperturbed system will be described, then...


\section{Methods and theory} \label{methods_and_theory}

\section{Results} \label{results}

\section{Discussion} \label{discussion}

\section{Conclusion} \label{conclusion}

\section{Appendix} \label{appendix}

\printbibliography

\end{document}
